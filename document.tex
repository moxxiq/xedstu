% ******** Приклад оформлення документа за ДСТУ 3008-95 ********
% ******************** автор: Тавров Д. Ю. *********************

% зазначаємо стильовий файл, який будемо використовувати
\documentclass{xedstu}

% починаємо верстку документа
\begin{document}

% створимо титульний аркуш
% за допомогою спеціальної команди
% \maketitlepage{params},
% де params --- це розділені комами пари "параметр={значення}"
\maketitlepage{
% StudentName --- прізвище, ініціали студента
	StudentName={Іваненко І. І.},
% StudentMale --- стать студента (true, якщо чоловік, false --- якщо жінка)
	StudentMale=true,
% StudentGroup --- група студента
	StudentGroup={КМ-00},
% Title --- назва
	Title={Звіт\\із лабораторної роботи №X\\із дисципліни \invcommas{Навчальна дисципліна}\\на тему\\Тема лабораторної роботи},
% SupervisorDegree --- науковий ступінь, учене звання керівника роботи
% якщо наукового ступеня немає, можна відповідний рядочок пропустити
	SupervisorDegree={канд. техн. наук, доцент},
% SupervisorName --- прізвище, ініціали керівника роботи
	SupervisorName={Петренко П. П.}
}

% створюємо анотацію
\abstractUkr
% далі пишемо текст анотації
Текст анотації.

% створюємо анотацію англійською мовою
\abstractEng
% далі пишемо текст анотації англійською мовою
Abstract text.

% створюємо зміст
\tableofcontents

% створюємо перелік умовних позначень, скорочень і термінів
\shortings
ПМА --- прикладна математика.

% створюємо вступ
\intro
Цей шаблон потрібно використовувати під час підготовки документації на кафедрі прикладної математики (ПМА). 

Текст вступу.

Текст вступу.

% створюємо мету
\goal
Метою роботи є створення шаблону оформлення документації в \XeLaTeX

% створюємо перший розділ роботи
\chapter{Назва першого розділу}
\label{chap:first}

Текст першого розділу:
% створюємо нумерований список
\begin{enumerate}
	\item перший елемент нумерованого списку;
	\item другий елемент нумерованого списку;
	\item тощо.
\end{enumerate}

Текст першого розділу:
% створюємо ненумерований список
\begin{itemize}
	\item перший елемент ненумерованого списку;
	\item другий елемент ненумерованого списку;
	\item тощо.
\end{itemize}

Текст першого розділу:
% створюємо нумерований список з двома рівнями вкладення
\begin{enumerate}
	\item перший елемент нумерованого списку;
	\item другий елемент нумерованого списку;
	\begin{enumerate}
		\item перший елемент другого рівня вкладення;
		\item другий елемент другого рівня вкладення;
	\end{enumerate}
	\item тощо.
\end{enumerate}

% створюємо другий розділ
\chapter{Назва другого розділу}
\label{chap:second}

% створюємо підрозділ
\section{Назва першого підрозділу другого розділу}
\label{sec:secondfirst}

Текст підрозділу.

% створюємо формулу
Формула:
\begin{equation}
x = y + z\;,
\label{eq:equation1}
\end{equation}
де $x$ --- це [...];

$y$ --- це [...];

$z$ --- це [...].

\section{Назва другого підрозділу другого розділу}
\label{sec:secondsecond}

% створюємо пункт
\subsection{Назва першого пункта другого підрозділу другого розділу}
\label{subsec:secondsecondfirst}

Текст пункта.

% приклади посилань на:
% розділ
Як було зазначено в розділі \ref{chap:first}, [...].

% підрозділ/пункт
Як наведено в \ref{sec:secondfirst}, [...].

% формулу
У (\ref{eq:equation1}) наведено [...].

% літературне джерело
Як зазначено в \cite{SSA}, [...].

% рисунок
[...] представлено на рисунку \ref{fig:figure1}.
	
% створюємо рисунок
\begin{figure}[!htp]
	\centering
	\includegraphics[scale=0.5]{PNG/figure1.png}
	\caption{Назва рисунка}
	\label{fig:figure1}
\end{figure}

% приклад посилання на таблицю
[...] представлено в таблиці \ref{table:table1}.
	
% таблиця
\begin{table}{|c|l|l|}{Назва таблиці}{table:table1}
	{\hline
	{\centering Назва першого стовпця} & {\centering Назва другого стовпця} & {\centering Назва третього стовпця} \\
	\hline}
	1 & Текст & Текст\\
	\hline
	1 & Текст & Текст\\
	\hline 
	1 & Текст & Текст\\
	\hline 
	1 & Текст & Текст\\
	\hline 
	1 & Текст & Текст\\
	\hline 
	1 & Текст & Текст\\
	\hline 
	1 & Текст & Текст\\
	\hline 
	1 & Текст & Текст\\
	\hline 
	1 & Текст & Текст\\
	\hline 
	1 & Текст & Текст\\
	\hline 
	1 & Текст & Текст\\
	\hline 
	1 & Текст & Текст\\
	\hline 
	1 & Текст & Текст\\
	\hline 
	1 & Текст & Текст\\
	\hline 
	1 & Текст & Текст\\
	\hline 
	1 & Текст & Текст\\
	\hline 
	1 & Текст & Текст\\
\end{table}

Текст розділу.

% створюємо Висновки
\conclusions

Цей шаблон потрібно використовувати під час підготовки документації на кафедрі ПМА. У ньому враховано вимоги кафедри ПМА, які уточнюють положення ДСТУ 3008-95 \invcommas{Документація. Звіти у сфері науки і техніки. Структура і правила оформлення}.

Текст висновків.

Текст висновків.
	
% створюємо перелік посилань
\begin{thebibliography}
	\bibitem{Mallat} Mallat S. A Wavelet Tour of Signal Processing / S. Mallat. --- New York: Academic Press, 1999. --- 620 p.
	
	\bibitem{Davydov} Давыдов А. А. Вейвлет-анализ социальных процессов / А. А. Давыдов // Социологические исследования. --- 2003. --- №11. --- С. 89--101.
	
	\bibitem{Zadeh} Zadeh L. A. The Concept of a Linguistic Variable and its Application to Approximate Reasoning / L. A. Zadeh // Information Sciences. --- 1975. --- 8. --- P. 199--249.
	
	\bibitem{Olivetti} Olivetti E. Induction in neuroscience with classification: issues and solutions / E.~Olivetti, S. Greiner, P. Avesani // Machine Learning and Interpretation in Neuroimaging [ed.~G. Langs, I. Rish, M. Gross-Wentrup, B. Murphy]. --- Berlin, Heidelberg : Springer-Verlag, 2012. --- P. 42--50. --- (Lecture Notes in Computer Science, vol. 7263.)
	
	\bibitem{Dawkins} Dawkins R. The Selfish Gene / Richard Dawkins. --- [3rd ed.]. --- Oxford, New York : Oxford University Press, 2006. --- 360 p.
	
	\bibitem{SSA} Голяндина Н. Э. Метод \invcommas{Гусеница}-SSA: анализ временных рядов : [учеб. пособие] / Н. Э. Голяндина. --- СПб., 2004. --- 76 с.
	
	\bibitem{standard} Information Technology --- Security Techniques --- Evaluation Criteria for IT Security --- Part 2: Security Functional Components : ISO/IEC 15408-2:2008. --- [Чинний від 2008-08-15]. --- 2008. --- 218 p.
	
	\bibitem{url} Base Structure Report (A Summary of DoD's Real Property Inventory) Fiscal Year 2001 Baseline [Electronic resource] / Office of the Deputy under Secretary of Defense.~--- 2000. --- Mode of access: 
\url{http://archive.defense.gov/news/Aug2001/basestructure2001.pdf}
\end{thebibliography}

% створюємо додаток
\append{Назва додатку}

% кожний лістинг вставляється в додаток за допомогою спеціальної команди,
% перший аргумент якої --- це заголовок, який з'являтиметься в тексті,
% другий --- шлях до файлу з лістингом
\listing{code.cpp}{SRC/code.cpp}

\append{Python файл}
\listingverbatim{hello.py}{SRC/hello.py}

\end{document}
